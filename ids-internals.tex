\documentclass[paper=a4]{scrartcl}
\usepackage[utf8]{inputenc}
\usepackage[T1]{fontenc}
\usepackage{graphicx}
\usepackage{url}

\title{Overview of IDS Internals}
\author{Rolf Krahl}

% Add a revision hint as a unnumbered footnote
\newcommand{\revhint}{%
  \begingroup%
  \let\thefootnote\relax%
  \footnote{Revision: \input{.revision}}%
  \addtocounter{footnote}{-1}%
  \endgroup%
}

\graphicspath{{../Abbildungen/}}

\begin{document}

\maketitle

\section{Introduction}

\revhint{}%
The ICAT Data Service (IDS) operates with delayed actions, internal
states, and multiple threads.  This makes it somewhat difficult to
understand from mere reading the sources what actions may be performed
under certain conditions.  A detailed knowlegde of the internal
processes of IDS may be needed to properly implement a storage plugin
in a non-trivial setup, in particular the context that each of the
plugin's methods may be called in.  This text shall provide a
reference to accommodate this need.  It is however restricted to the
case that the IDS is configured as a two level storage and the storage
unit is dataset.

The workflow in the IDS can be sketched as follows: the IDS waits for
incoming user requests coming over the RESTful interface.  In some
cases, these requests may be completed immediately.  In other cases,
the request only queue a deferred operation that will be processed in
the background later.  Section \ref{sec:requests} lists the user
request and what is done in each case.  The various operations that
are queued for any given dataset are kept track of in a finite state
machine.  This is detailed in Section \ref{sec:fsm}.  When processing
the queue, a new thread is started that performs the operation.  What
each of the deferred operations do is described in Section
\ref{sec:defops}.


\section{User requests}
\label{sec:requests}

\ldots


\section{Finite state machine}
\label{sec:fsm}

\ldots


\section{Deferred operations}
\label{sec:defops}

\ldots



\end{document}


